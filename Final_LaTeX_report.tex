\documentclass{article}
\usepackage{lipsum} 

\title{GenAI in Database Design}
\author{Praneetha Praveen Kalbhavi}
\date{\today}

\begin{document}

\maketitle

\section{Introduction}
GenAI, an AI language model, played a crucial role in designing a Pokémon database. We explored both SQL and NoSQL approaches to model the data. Let's dive into the technical details and our experiences:

\section{SQL Database Design}

\subsection{Pokémon Table}
The heart of our SQL schema lies in the Pokémon table. GenAI provided us with SQL queries to create this table. Here's how we structured it:

\begin{itemize}
    \item \textbf{pokemon\_id} (Primary Key): A unique identifier for each Pokémon.
    \item \textbf{pokemon\_name}: The name of the Pokémon.
    \item \textbf{primary\_type}: Represents the primary type of the Pokémon (e.g., Grass, Fire, Water).
    \item \textbf{secondary\_type} (nullable): Some Pokémon have a secondary type, enhancing their versatility.
\end{itemize}

\subsection{Types Table}
To capture the essence of Pokémon types, we created a separate table:

\begin{itemize}
    \item \textbf{type\_id} (Primary Key): Uniquely identifies each Pokémon type.
    \item \textbf{type\_name}: Describes the Pokémon type (e.g., Electric, Psychic, Fighting).
\end{itemize}

\subsection{Moves Table}
Moves play a pivotal role in Pokémon battles. GenAI's assistance was invaluable in designing this table:

\begin{itemize}
    \item \textbf{move\_id} (Primary Key): A unique identifier for each move.
    \item \textbf{move\_name}: The name of the move.
    \item \textbf{power}: Power value associated with the move.
    \item \textbf{type\_id} (Foreign Key referencing Types Table): The type of the move.
\end{itemize}

\section{NoSQL Considerations (Using MongoDB)}
When discussing NoSQL databases, GenAI emphasized flexibility. We chose MongoDB as our NoSQL solution. Here's how we adapted our design:

\subsection{Pokémon Document}
Instead of rigid tables, we stored Pokémon and Moves as documents in collections. Each Pokémon document contains information about its types and moves:

\begin{verbatim}
{
  "_id": "unique_pokemon_id",
  "pokemon_name": "Bulbasaur",
  "types": ["Grass"],
  "moves": ["Tackle", "Vine Whip", "Return"]
}
\end{verbatim}

\subsection{Moves Document}
Similarly, moves are represented as documents:

\begin{verbatim}
{
  "_id": "unique_move_id",
  "move_name": "Tackle",
  "power": 35,
  "type": "Normal"
}
\end{verbatim}

\section{Conclusion}
Our journey with GenAI was enlightening. Its versatility in handling both SQL and NoSQL scenarios made the database design process efficient and insightful.

\end{document}
